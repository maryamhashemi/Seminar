% !TeX root=main.tex
% در این فایل، عنوان پایان‌نامه، مشخصات خود و چکیده پایان‌نامه را به انگلیسی، وارد کنید.

%%%%%%%%%%%%%%%%%%%%%%%%%%%%%%%%%%%%
\baselineskip=.6cm
\begin{latin}
\latinuniversity{Iran University of Science and Technology}
\latinfaculty{Computer Engineering Department}
\latinsubject{Computer Engineering }
\latinfield{Artificial Intelligence}
\latintitle{Visual Question Answering}
\firstlatinsupervisor{Dr. Sayyed Sauleh Eetemadi}

\latinname{Maryam Sadat}
\latinsurname{Hashemi}
\latinthesisdate{December 2020}
\latinkeywords{Visual Question Answering, Natural Language Processing, Computer Vision, Deep Learning, pretrained models}
\en-abstract{
Visual Question Answering(VQA) is a challenging task that has been introduced in recent years and has received increasing attention from both the computer vision and the natural language processing communities. Visual Question Answering aims to answer the questions about given images. A VQA system tries to find the correct answer to questions using visual elements of the image and inference gathered from textual questions. In the first chapter of this review, we present the Visual Question Answering task, applications, and challenges. After defining some concepts in the second chapter, we discuss various datasets for VQA, methods, and evaluation metrics in chapter 3. Due to the success of deep learning and pre-trained models, we classify VQA methods into two general approaches: deep learning and pre-trained models. In the last chapter, after concluding on the different aspects of VQA, we provide some directions for future work.
}
\latinfirstPage
\end{latin}
