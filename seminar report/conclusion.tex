% !TeX root=main.tex
\chapter{نتیجه‌گیری و کار‌های آینده}
\thispagestyle{empty}

\section{نتیجه‌گیری}
	 علی رغم این که از معرفی مسئله پرسش و پاسخ تصویری تنها چندین سال می‌گذرد، رشد آن در این چند سال قابل توجه‌ بوده است. مجموعه‌دادگان بسیاری با اهداف مختلف در طی این سال‌ها معرفی شد. برای حل مسئله پرسش و پاسخ تصویری، رویکرد‌های یادگیری عمیق همچنان در مرکز توجه هستند. ما برجسته‌ترین مدل‌های یادگیری عمیق برای مسئله پرسش و پاسخ تصویری را بررسی کردیم. با معرفی شبکه‌های از قبل آموزش‌دیده، بهبود چشمگیری در مسائل یادگیری عمیق رخ داد به طوری که بیشتر مسائل مختلف در یادگیری عمیق، بهترین نتیجه خود را با استفاده از شبکه‌های از قبل آموزش‌دیده بدست آورده‌اند. مسئله پرسش و پاسخ تصویری نیز از این قاعده مستثنی نیست و در حال حاضر شبکه‌های از قبل آموزش‌دیده بر روی زبان طبیعی و تصویر بهترین عملکرد را برای مجموعه‌دادگان پرسش و پاسخ تصویری رقم زده‌اند. چندین نمونه از این مدل‌ها را با جزئیات بحث کردیم. در آخر معیارهایی را معرفی کردیم که بتوان با آن‌ها مدل‌های پرسش و پاسخ تصویری را ارزیابی کرد. البته که ارزیابی مسئله پرسش و پاسخ تصویری همچنان یک مسئله حل نشده است و نیاز به تحقیقات بیشتری دارد. پیشرفت‌های زیادی که همچنان برای مجموعه‌دادگان مختلف در این حوزه اتفاق می‌افتد، به این معناست که هنوز فضای زیادی برای نوآوری در آینده وجود دارد.
\section{مسائل باز و کارهای قابل انجام}
با وجود تمام پیشرفت‌هایی که در سال‌های اخیر در مسئله پرسش و پاسخ تصویری اتفاق افتاده است، مدل‌های پیشنهاد شده در این حوزه با نواقصی مواجه هستند. اولین مشکل روش‌های فعلی پاسخ به سوالاتی است که نیاز به استدلال طولانی دارند. از طرفی منبع بهبود‌های نسبی مدل‌های موجود واضح نیست و مشخص نیست که مدل تا چه اندازه مفاهیم مشترک بین زبان و تصویر را درک می‌کند و چگونه از پیوند این دو برای پیش‌بینی پاسخ استفاده می‌کند. پس اگر بتوانیم بفهمیم که روند درک مدل‌هایی فعلی از زبان و تصویر چگونه است، می‌توانیم مدلی را پیشنهاد دهیم که بتواند به سوالاتی که نیاز به استدلال طولانی دارند، پاسخ دهد.

اکثر روش‌های پیشنهادشده، مسئله پرسش و پاسخ تصویری را یک مسئله ‌طبقه‌بندی در نظر می‌گیرند و تعداد کمی از کارهای انجام شده به دنبال تولید پاسخ بوده‌اند. یکی از دلایلی که باعث کم توجهی به تولید پاسخ شده است، زمان‌بر بودن فرآیند آن است. یکی از راه‌حل‌های این مشکل می‌تواند استفاده از ترنسفرمرها با چندین لایه رمزگذار و رمزگشا بر روی هم باشد. از معماری ترنسفرمر  برای تولید پاسخ در پرسش و پاسخ تصویری به صورت محدود استفاده شده است. از طرفی، موفقیت ترنسفرمر‌ها در مسائل پردازش زبان طبیعی، ما را ترغیب می‌کند که از قدرت آن‌ها در مسئله پرسش و پاسخ تصویری برای تولید پاسخ در آینده استفاده کنیم.

یکی دیگر از محدودیت‌های مسئله پرسش و پاسخ تصویری، فقدان مجموعه‌دادگان متناسب با واقعیت است. در حال حاضر نمی‌توان از دادگان موجود در مسئله پرسش و پاسخ تصویری برای کاربرد‌های عملی مانند کمک به افراد نابینا و کم‌بینا استفاده کرد. از طرف دیگر اکثر مجموعه‌دادگان با مشکل بایاس مواجه هستند. بنابراین جمع‌آوری و تهیه مجموعه‌دادگانی که منطبق با کاربرد عملی در جامعه و بدون بایاس باشند، اهمیت پیدا می‌کند.

در فصل قبل دیدیم که در حال حاضر، بهترین عملکرد برای مجموعه‌دادگان پرسش و پاسخ تصویری توسط شبکه‌های از قبل آموزش‌دیده بر روی زبان طبیعی و تصویر بدست آمده است. اساس و پایه‌ی این شبکه‌ها، ترنسفرمر است. یکی از بزرگترین مشکلات ترنسفرمرها این است که محاسبه توجه از مرتبه زمانی و حافظه‌ای 2 است. اخیراً روش‌های زیادی مانند
\lr{Reformer}\cite{kitaev2020reformer}
و
\lr{Performer}\cite{performer}
پیشنهاد شده است که مرتبه زمانی و حافظه‌ای ترنسفرمرها را کاهش می‌دهند. بنابراین یکی از مسیرهای تحقیقاتی پیش رو، استفاده از این ترنسفرمر‌های بهبودیافته در معماری شبکه‌های از قبل آموزش دیده بر روی زبان طبیعی و تصویر می‌تواند باشد.

با توجه به دانشی که ما بدست آوردیم، تاکنون هیچ‌گونه تحقیقی در مورد پرسش و پاسخ تصویری
در زبان فارسی انجام نشده است. از این رو دادگان مناسبی نیز برای این کار وجود ندارد. پس تهیه و جمع‌آوری دادگان فارسی برای مسئله پرسش و پاسخ تصویری و آموزش یک مدل کارآمد براساس آن، یک کار ارزشمند خواهد بود و مسیر جدیدی را برای سایر محققین باز خواهد کرد.


