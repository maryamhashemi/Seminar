% !TeX root=main.tex

\university{علم و صنعت ایران}

\faculty{دانشکده مهندسی کامپیوتر}

\department{گروه هوش مصنوعی}

\subject{مهندسی کامپیوتر}

\field{هوش مصنوعی}

\title{پرسش و پاسخ تصویری}

\firstsupervisor{دکتر سید صالح اعتمادی}

\name{مریم سادات}

\surname{هاشمی}

\studentID{98722233}

\thesisdate{دی 1399}

\projectLabel{گزارش سمینار}

\firstPage
\besmPage

\keywords{پرسش و پاسخ تصویری، پردازش زبان طبیعی، بینایی ماشین، یادگیری عمیق، مدل‌های از قبل آموزش‌دیده}
\fa-abstract{
مسئله پرسش و پاسخ تصویری یک مسئله چالش برانگیز است که در سال‌‌های اخیر معرفی شده است و مورد توجه بسیاری از محققان دو حوزه پردازش زبان طبیعی و بینایی ماشین قرار گرفته است. هدف این مسئله پاسخ به پرسش مطرح شده در مورد تصویر ورودی است. یک سیستم پرسش و پاسخ تصویری سعی می‌کند با استفاده از عناصر بصری تصویر و استنتاج جمع‌آوری شده از سوال متنی، پاسخ صحیح را پیدا کند. در فصل اول این بررسی، به معرفی مسئله پرسش و پاسخ تصویری، کاربرد و اهمیت آن و چالش‌های این مسئله می‌پردازیم. پس از تعریف  برخی مفاهیم مورد نیاز در فصل دوم، مجموعه‌دادگان، روش‌های حل مسئله پرسش و پاسخ و تصویری و معیارهای ارزیابی آن را در فصل سوم بررسی می‌کنیم. با توجه به موفقیت ‌یادگیری عمیق و مدل‌های از قبل آموزش دیده، رویکرد‌های حل مسئله پرسش و پاسخ تصویری را به دو دسته کلی رویکرد‌ یادگیری عمیق و رویکرد‌ مدل‌های از قبل آموزش‌‌دیده نقسیم‌بندی می‌کنیم. در فصل آخر، پس از نتیجه‌گیری در مورد ابعاد مختلف مسئله پرسش و پاسخ تصویری، در مورد مسیرهای تحقیق در آینده بحث می‌کنیم.
}
\abstractPage

\newpage
\clearpage