% !TeX root=main.tex

\pagenumbering{arabic}

\chapter{}
%\thispagestyle{empty}
\section{بررسی کلی و تعریف مبحث پرسش و پاسخ تصویری}
باید تکمیل شود.
\section{کاربرد و اهمیت این مسئله}
باید تکمیل شود.
\section{بررسی چالشهای موجود در این مسئله}
باید تکمیل شود.
\section{بررسی مجموعه دادگان مطرح و مسابقات مطرح این حوزه}
باید تکمیل شود.
\subsection{مجموعه داده DAQUAR}
{
	\textbf{DAQUAR}
	 مخفف
	\lr{Dataset for Question Answering on Real World Images}
است که توسط مالینوفسکی منتشر‌شده‌است. این اولین مجموعه‌داده‌ای است که برای مسئله 
	\lr{VQA}
 منتشر‌شده‌است. تصاویر از مجموعه‌داده
  \lr{NYU-Depth V2}
   گرفته‌شده‌است.  اندازه این مجموعه‌داده کوچک است و در مجموع 1449 تصویر دارد. 
 DAQUAR
 شامل \textit{12468} زوج پرسش و پاسخ با 2483 سوال منحصربه‌فرد است. برای تولید پرسش و پاسخ‌ها از دو روش مصنوعی و انسانی استفاده‌شده‌است. در روش مصنوعی پرسش و پاسخ‌ها به صورت خودکار از الگوهای موجود در جدول فلان تولید‌شده‌است. در روش دیگر از 5 نفر انسان خواسته‌شده‌است تا پرسش و پاسخ تولید کنند. تعداد پرسش و پاسخ‌های آموزشی در این مجموعه‌داده 6794 و تعداد پرسش و پاسخ‌های تست 564 است و به طور میانگین برای هر عکس تقریبا 9 پرسش و پاسخ وجود دارد. این مجموعه‌داده با مشکل بایاس روبه‌رو است زیرا تصاویر این مجموعه تنها مربوط به داخل خانه است و بیش از 400 مورد وجود دارد که اشیایی مثل میز و صندلی در پاسخ‌ها تکرارشده‌است.
}
\subsection{مجموعه داده VQA}
باید تکمیل شود.
\subsection{مجموعه داده Visual Madlibs}
باید تکمیل شود.
\subsection{مجموعه داده Visual 7w}
باید تکمیل شود.
\subsection{مجموعه داده CLEVR}
باید تکمیل شود.
\subsection{مجموعه داده Tally-QA}
باید تکمیل شود.
\subsection{مجموعه داده KVQA}
باید تکمیل شود.
\section{بررسی فازهای مختلف مسئله پرسش و پاسخ تصویری}
باید تکمیل شود.
\subsection{فاز 1 : استخراج ویژگی از تصویر و سوا ل}
باید تکمیل شود.
\subsection{فاز 2 : درک مشترک تصویر و سوا ل}
باید تکمیل شود.
\subsection{فاز 3 : تولید جواب}
باید تکمیل شود.
\section{معیارهای ارزیابی مسئله پرسش و پاسخ تصویری}
باید تکمیل شود.
\section{ چگونگی ساخت مجموعه داده حاوی پرسش و پاسخ به زبان فارسی}
باید تکمیل شود.