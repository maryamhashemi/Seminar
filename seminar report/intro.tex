% !TeX root=main.tex

\pagenumbering{arabic}

\chapter{}
%\thispagestyle{empty}
\section{بررسی کلی و تعریف مبحث پرسش و پاسخ تصویری}
باید تکمیل شود.
\section{کاربرد و اهمیت این مسئله}
باید تکمیل شود.
\section{بررسی چالشهای موجود در این مسئله}
باید تکمیل شود.
\section{بررسی مجموعه دادگان مطرح و مسابقات مطرح این حوزه}
باید تکمیل شود.
\subsection{مجموعه داده DAQUAR}
{
	\textbf{DAQUAR}
	 مخفف
	\lr{Dataset for Question Answering on Real World Images}
است که توسط مالینوفسکی منتشر‌شده‌است. این اولین مجموعه‌داده‌ای است که برای مسئله 
	\lr{VQA}
 منتشر‌شده‌است. تصاویر از مجموعه‌داده
  \lr{NYU-Depth V2}
   گرفته‌شده‌است.  اندازه این مجموعه‌داده کوچک است و در مجموع 1449 تصویر دارد. 
 DAQUAR
 شامل \textit{12468} زوج پرسش و پاسخ با 2483 سوال منحصربه‌فرد است. برای تولید پرسش و پاسخ‌ها از دو روش مصنوعی و انسانی استفاده‌شده‌است. در روش مصنوعی پرسش و پاسخ‌ها به صورت خودکار از الگوهای موجود در جدول فلان تولید‌شده‌است. در روش دیگر از 5 نفر انسان خواسته‌شده‌است تا پرسش و پاسخ تولید کنند. تعداد پرسش و پاسخ‌های آموزشی در این مجموعه‌داده 6794 و تعداد پرسش و پاسخ‌های تست 564 است و به طور میانگین برای هر عکس تقریبا 9 پرسش و پاسخ وجود دارد. این مجموعه‌داده با مشکل بایاس روبه‌رو است زیرا تصاویر این مجموعه تنها مربوط به داخل خانه است و بیش از 400 مورد وجود دارد که اشیایی مثل میز و صندلی در پاسخ‌ها تکرارشده‌است.
}
\subsection{مجموعه داده VQA}
{
مجموعه‌داده
 \lr{Visual Question Answering v1(VQA v1)}
یکی از پرکاربردترین مجموعه‌داده‌ها در زمینه پرسش و پاسخ تصویری است. این مجموعه‌داده شامل دو بخش است. یک بخش از تصاویر واقعی ساخته‌شده‌است که
 \lr{VQA-real} 
 نام‌دارد و دیگری با تصاویر کارتونی ساخته‌شده‌است که با نام 
 \lr{VQA-abstract}
 از آن در مقالات یاد می‌شود.
 
 
 \lr{VQA-real} 
 به ترتیب شامل 123287 تصویر آموزشی و 81434 تصویر آزمایشی است که این تصاویر از مجموعه‌داده
 \lr{MS-COCO}
  تهیه شده است.  برای جمع‌آوری پرسش و پاسخ هم از نیروی انسانی استفاده‌شده‌است. برای هر تصویر حداقل 3 سوال منحصربه‌فرد وجود دارد و برای هر سوال 10 پاسخ توسط کاربرهای غیرتکراری جمع‌آوری‌شده‌است. این مجموعه‌داده شامل 614163 سوال به صورت 
  \lr{open-ended}
  و چندگزینه‌ای است. در (اشاره به مقاله) بررسی دقیقی در مورد نوع سوالات، طول سوالات و پاسخ‌ها و غیره انجام‌شده‌است.
  
  
 \lr{VQA-abstract}
 به عنوان یک مجموعه‌داده جداگانه و مکمل در کنار
 \lr{VQA-real}
 قرار دارد. هدف از این مجموعه‌داده از بین بردن نیاز به تجزیه و تحلیل تصاویر واقعی است تا مدل‌ها برای پاسخ به سوالات تمرکز خود را بر روی استدلال‌های سطح بالاتری بگذارند. تصاویر کارتونی در این مجموعه‌داده به صورت دستی توسط انسان‌ها و به وسیله‌ی رابط کاربری که از قبل آماده‌شده‌است؛ ساخته‌شده‌است. تصاویر می‌تواند دو حالت را نشان‌دهند: داخل خانه و خارج از خانه که هر کدام مجموعه متفاوتی از عناصر را شامل می‌شوند از جمله حیوانات، اشیا و انسان‌ها با حالت‌های مختلف. در مجموع 50000 تصویر ایجاد‌شده‌است. مشابه تصاویر واقعی 3 سوال برای هر تصویر (یعنی در کل 150000 سوال) و برای هر سوال 10 پاسخ  جمع‌آوری‌شده‌است.
 
 مجموعه‌داده 
\lr{Visual Question Answering v2(VQA v2)}
در سال 2017 پس از مجموعه‌داده 
\lr{VQA v1}
معرفی شد. 
\lr{VQA v2}
نسبت به 
\lr{VQA v1}
متوازن تر است و تعصبات زبانی در 
\lr{VQA v1}
را کاهش داده است. اندازه‌ی مجموعه داده‌ی
\lr{VQA v2}
تقریبا دو برابر مجموعه‌داده‌ی 
\lr{VQA v1}
است. در مجموعه‌داده‌ی
\lr{VQA v2}
تقریبا برای هر سوال دو تصویر مشابه وجود دارد که پاسخ‌های متفاوتی برای سوال دارند.
}

\subsection{مجموعه داده Visual Madlibs}
باید تکمیل شود.
\subsection{مجموعه داده Visual 7w}
باید تکمیل شود.
\subsection{مجموعه داده CLEVR}
باید تکمیل شود.
\subsection{مجموعه داده Tally-QA}
باید تکمیل شود.
\subsection{مجموعه داده KVQA}
باید تکمیل شود.
\section{بررسی فازهای مختلف مسئله پرسش و پاسخ تصویری}
باید تکمیل شود.
\subsection{فاز 1 : استخراج ویژگی از تصویر و سوا ل}
باید تکمیل شود.
\subsection{فاز 2 : درک مشترک تصویر و سوا ل}
باید تکمیل شود.
\subsection{فاز 3 : تولید جواب}
باید تکمیل شود.
\section{معیارهای ارزیابی مسئله پرسش و پاسخ تصویری}
باید تکمیل شود.
\section{ چگونگی ساخت مجموعه داده حاوی پرسش و پاسخ به زبان فارسی}
باید تکمیل شود.